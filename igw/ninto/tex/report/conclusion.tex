\section{Élements de conclusion}

\subsection{Résumé}
Une double étude mono-paramétrique modale des IGW atmosphériques a été menée, dans la perspective d'obtenir un code de simulation numérique où les différents paramètres importants pour la propagation 3D seraient intégrés. Au cours de ce stage, l'effet des vents et de la viscosité a été abordé. Le rôle des vents moyens comme filtre directionnel a été reproduit, en accord avec \cite{Sun2007} ; l'effet d'amplification/atténuation relative a par la suité été appliqué à un système d'IGW tsunamigénique, associé à un profil réaliste des vents atmosphériques. Une amplification importante est attendue pour les IGW se propageant contre le vent moyen.
% TODO donner le résultat essentiel pour le cas réaliste : qui quoi comment ?
Le rôle de la viscosité a été exploré analytiquement et les bases numériques ont posées, quoique l'implémentation effective n'est pas été achevée. L'adimensionnement du système d'étude s'est révélé nécessaire pour à terme intégrer numériquement les ordres de grandeur des différentes inconnues du problème.

\subsection{Bilan personnel du stage}

Je souhaite tout d'abord remercier Giovanni Occhipinti. Après avoir défini les objectifs généraux à atteindre, il m'a laissé toute latitude pour me familiariser avec mon sujet et l'aborder d'une manière ouverte. Nous avons peu communiqué et j'ai réalisé un travail indépendant -- sûrement trop ! Cette méthodologie présente des risques évidents de découragement et d'égarement, en partie expérimentés ; en contre-partie, cela m'a amené à prendre à bras le corps un sujet transverse et à réaliser un travail le plus cohérent possible. J'ai progressé analytiquement (en explorant les aspects rhéologiques pour contrôler un résultat qui me semblait douteux, et en envisageant — malheureusement un peu tard — l'adimensionnement de mon système d'étude) ; j'ai progressé numériquement (avec le soucis de faire un code vectoriel efficace, ce qui a été plus payant quoique les simulations aient été effectuées à basse résolution !).

Des notions importantes m'ont manqué en cours de route et je n'ai pas toujours, à tort, cherché de l'aide auprès de mon tuteur, souhaitant me « débrouiller » autant que possible par moi-même — erreur ! Le fait de travailler sur deux paramètres m'a amené progressivement délaisser l'un au profit de l'autre, \ie à privilégier le numérique sur l'analytique. Ayant gagné une meilleure compréhension des aspects adimensionnels dans les derniers jours du stage, la perspective d'achever l'étude et de continuer au-delà est réjouissante. Il s'agissait de ma toute première année universitaire en physique et ma motivation s'est trouvée renforcée par ce stage.

Évidemment, « coder » en amont du traitement de résultats est une tâche relativement solitaire et propice au surplace (bugs, nécessité d'apprendre en cours de route des techniques numériques spécifiques, donc de s'informer, produire les tests en relation, etc.) Je souhaiterais par la suite développer un code adimensionnel efficace, dans lequel il sera possible d'intégrer des paramètres supplémentaires. Il y aura alors la perspective de sortir de l'étude purement modale pour passer à un traitement basé, par exemple, sur les relations de dispersion. On pourra alors envisager de traiter des données réelles (Sumatra-Andaman, problème inverse), de paralléliser le code, etc.

