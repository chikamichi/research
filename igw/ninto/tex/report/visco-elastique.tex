\changepage{-3cm}%amount added to textheight
           {}%amount added to textwidth
           {}%amount added to evensidemargin
           {}%amount added to oddsidemargin
           {}%amount added to columnsep
           {2cm}%amount added to topmargin
           {}%amount added to headheight
           {}%amount added to headsep
           {}%amount added to footskip
\section{Modèle visqueux}

À l'échelle de la propagation des IGW, le milieu atmosphèrique possède une viscosité faible mais non négligeable sous certaines conditions rencontrées notamment dans la haute atmosphère. La diffusion de quantité de mouvement qui lui est associée peut constituer un phénomène dissipatif de premier plan des ondes amplifiées, au coté de la diffusion thermique et de la turbulence. Les IGW développées par des sources de basses/moyennes altitudes (topographie, cisaillement de vents, convection…) transportent de l'énergie et de la quantité de mouvement des régions de haute densité vers des régions de plus faible densité. L'estimation de l'efficacité de leur dissipation est essentielle pour paramétrer cette redistribution énergétique : des simulations numériques en ont dressées les caractéristiques principales (voir par exemple \cite{Vadas2005}).

Afin d'introduire sans \emph{a priori} ce nouveau paramètre dans le modèle utilisé pour l'étude modale, une revue du problème sous un angle rhéologique (\cite{Gueguen1994}) a tout d'abord été effectuée, en délaissant en première approche la dérivation usuelle mais complexe d'une relation de dispersion à dépendance fréquentielle. Une implémentation numérique de la propagation visqueuse dans une atmosphère neutre a ensuite été entreprise.

\subsection{Dérivation analytique}

En partant de la forme locale générale de l'équation de la dynamique
\beq \quad \rho \left ( \delt \vv + \vv \cdot \gradient \vv \right ) = \divergence \Tctr + \vf \quad \label{eq:dynamique} \eeq

avec $\vf$ la densité volumique des forces extérieures et $\Tctr$ le tenseur des contraintes. Par définition, $tr(\Tctr) = -\bar \pression \cdot$ dim, avec dim = 3 en 3D ($\bar \pression$ est la pression thermodynamique, identifiée à la pression hydrostatique $\pression$ pour la plupart des cas d'étude). De ce fait, on peut écrire \beq \Tctrij = \Tctrijdev - \kro \Tctrmoyij \eeq en séparant la partie déviatorique de la partie moyenne (isotrope).
\plop
Soit $\vu$ le vecteur déplacement, \emph{i.e.} $\vu = \overrightarrow{XX'}$ avec $X$ et $X'$ les positions d'un même point avant et après déformation. On définit le tenseur des déformations $\Tdef$ (\emph{strain tensor}) comme la partie symétrique de la matrice du gradient de déplacement \beq \Tdefij = \Tdefexpansion \eeq On a $tr(\Tdef) = \kro \Tdefij = \frac{\Delta V}{V} = \divergence \vu$, et on définit $\Tdefmoy = \frac{tr(\Tdef)}{3}$. Dès lors \beq \Tdefijdev = \Tdefij - \Tdefmoy \eeq

Soit $\vv$ le vecteur vitesse. On définit le tenseur du taux de déformation $\Ttxdef$ (\emph{strain rate tensor}) comme la partie symétrique de la matrice du gradient de vitesse, \emph{i.e.} la dérivée temporelle de $\Tdef$ \beq \Ttxdefij = \Ttxdefexpansion \eeq On a $tr(\Ttxdef) = \kro \Ttxdefij = \divergence \vv$ et on définit $\Ttxdefmoy = \frac{tr(\Ttxdef)}{3}$. Dès lors \beq \Ttxdefijdev = \Ttxdefij - \Ttxdefmoy \eeq
\plop
En utilisant la même idée de séparation entre une partie moyenne isotrope et une partie déviatorique, on peut exprimer toute la complexité du lien (tenseur d'ordre 4) entre contrainte et déformation dans le cas isotrope au moyen de deux cœfficients : un pour la variation de volume, un pour le cisaillement.

La viscosité newtonienne isotrope est définie telle que
\begin{subnumcases}{\label{eq:visc}}
    \Tctrijdev & $=\quad 2 \visc ~   \Ttxdefijdev$             \label{eq:visc-un} \\
    \Tctrmoy   & $=\quad ~ \viscbulk \divergence \vv$         \label{eq:visc-deux}
\end{subnumcases}
et l'élasticité isotrope est définie telle que
\begin{subnumcases}{\label{eq:elas}}
    \Tctrijdev & $=\quad 2 G ~ \Tdefijdev$                    \label{eq:elas-un}\\
    \Tctrmoy   & $=\quad ~ K ~ \divergence \vu = -\pression$  \label{eq:elas-deux}
\end{subnumcases}

où $\visc$ est la viscosité dynamique (aussi dite primaire, en cisaillement, \emph{shear viscosity}) et $\viscbulk$ la viscosité de volume (\emph{bulk viscosity}), en pascal seconde ; $G$ est le module de cisaillement, $K$ le module de compressibilité, en pascal.
%(note : $\viscbulk$ est notée $K$ dans le pdf en italien sur la viscosité que tu m'as donné).

Pour les fluides, qui se caractérisent par $G = 0$ (résistance nulle au cisaillement), un modèle classique consiste à prendre Kelvin%
\footnote{%
    Modèle de Kelvin type « ressort et amortisseur en parallèle » :
    \begin{numcases}{}
        \Tdef^{\textrm{élastique}}     & $=\quad \Tdef^{\textrm{visqueux}}$ \notag \\
        \Tctr                         & $=\quad \Tctr^{\textrm{élastique}} + \Tctr^{\textrm{visqueux}}$ \notag
    \end{numcases}
}
en volume ($\Tctrmoy$) et Maxwell%
\footnote{%
    Modèle de Maxwell type « ressort et amortisseur en série » :
    \begin{numcases}{}
        \Tctr^{\textrm{élastique}}     & $=\quad \Tctr^{\textrm{visqueux}}$ \notag \\
        \Ttxdef                     & $=\quad \Tdef^{\textrm{élastique}} + \Tdef^{\textrm{visqueux}}$ \notag
    \end{numcases}
} en cisaillement ($\Ttxdefdev$). Cela correspond à un comportement visco-élastique où la pression intègre toute la composante élastique. Si les forces volumiques dérivent d'un potentiel gravitationnel, \refeq{eq:dynamique} s'écrit
\begin{align}
\rho \dt{v_i}
  & = \deli{\Tctrij}{x_j} + \rho g_i \notag \\
\rho \dt \vv
  & = \divergence \Tctrdev + \divergence \overbrace{(- \pression + \viscbulk \divergence \vv)}^{\Tctrmoy} + \rho \vg \notag
  \intertext{et comme $G = 0$ pour les fluides, la partie déviatorique est purement visqueuse}
  & = \deli{2 \visc \Ttxdefijdev}{x_j} + \gradient \lp - \pression + \viscbulk \divergence \vv \rp + \rho \vg \notag \\
  \intertext{On montre que $2 \visc \divergence \Ttxdefdev = \visc \laplacien \vv$ (à $\visc$ constante, ce qui n'est pas irréaliste pour une application atmosphérique et facilite l'intégration numérique) et ce que l'hypothèse $\divergence \vv = 0$ soit appliquée ou non. De ce fait}
  \rho \dt{v_i} & = \visc \laplacien \vv + \gradient \lp - \pression + \viscbulk \divergence \vv \rp + \rho \vg \label{eq:dynamique-jd}
\end{align}

Par ailleurs, une expression du $\Tctrij$ visco-élastique \emph{pour les fluides newtoniens isotropes} est également donnée par une loi de comportement de Hooke mise sous la forme
\beq \Tctrij = - \kro \pression + \lambda \kro \Ttxdefkk + 2 \visc \Ttxdefij \eeq
où interviennent les cœfficients de Lamé : $\visc$ caractérise la résistance au cisaillement et $\lambda$ la compressiblité (constants pour un milieu homogène). On peut en déduire que la contrainte moyenne normale totale (viscosité + élasticité) vaut
\beq \Tctrmoy = \frac{\Tctrkk}{3} = -\pression + \left ( \lambda + \frac{2}{3} \visc \right ) \Ttxdefkk \eeq
donc \beq \viscbulk = \viscbulkexpansion \eeq
On peut dès lors utiliser cette formule pour calculer d'une autre manière le terme $\divergence \Tctrdev$ intervenant dans la forme locale \refeq{eq:dynamique}. On a
\begin{align}
\divergence \Tctrdev
  & = \lambda \gradient ( \divergence \vv) + 2 \visc \divergence \Ttxdef \notag \\
  & = \lambda \gradient ( \divergence \vv) + 2 \visc \divergence \left ( \Ttxdefdev + \Ttxdefmoy \right ) \notag \\
  & = \lambda \gradient ( \divergence \vv) + 2 \visc \divergence \Ttxdefdev + \frac{2\visc}{3} \gradient ( \divergence \vv ) \notag \\
  & = \underbrace{\lp \lambda + \frac{2\visc}{3} \rp}_{\viscbulk} \gradient ( \divergence \vv) + \visc \laplacien \vv \label{eq:dynamique-finale}
\end{align}
en cohérence avec \refeq{eq:dynamique-jd}.
  %; on peut éventuellement en déduire une forme rotationnelle}
  %& = \lambda \gradient ( \divergence \vv) + \visc \laplacien \vv - \visc \gradient ( \divergence \vv ) + \frac{5\visc}{3} \gradient ( \divergence \vv ) \notag \\
  %& = (\lambda + \frac{5\visc}{3} ) \gradient ( \divergence \vv) + \underbrace{\visc \laplacien \vv - \visc \gradient ( \divergence \vv )}_{- \visc \rotationnel(\rotationnel \vv)} \notag
Dans le cas d'un fluide de Stokes, on pose traditionnellement que $\viscbulk = \viscbulkexpansion = 0$.
%ce qui permet de simplifier l'expression ci-dessus en
%\beq (\lambda + \visc) \gradient ( \divergence \vv) - \visc \rotationnel(\rotationnel \vv) \label{eq:dynamique-lame} \eeq
Cette hypothèse n'est valide que pour les gazs monoatomiques et équivaut à poser que $\viscbulk = \frac{2 \visc(1 + \nu)}{3(1 - 2\nu)} = 0$ avec $\nu$ le cœfficient de Poisson, \ie $\nu = -1$ pour une viscosité primaire non nulle. Une telle valeur signifie que l'air est estimé être un milieu auxétique (le volume d'une parcelle réduit sous contrainte axiale par contraction dans la direction perpendiculaire à la contrainte). En réalité, l'air est un mélange gazeux. Toutefois, en appliquant à \refeq{eq:dynamique-jd} ou \refeq{eq:dynamique-finale} l'approximation de Boussinesq (impliquant $\divergence~\vv~=~0$), le terme en volume disparaît et on se ramène à la forme classique
\beq \rho \dt{\vv} = \visc \laplacien \vv - \gradient \pression + \rho \vg  \eeq

D'un point de vue numérique, conserver explicitement un terme en volume peut-être intéressant pour contrôler l'hypothèse $\divergence \vv = 0$ au regard des erreurs d'arrondis, en forçant sa nullité \emph{via} le champ de vitesse, si nécessaire. Physiquement parlant, cela revient toujours à poser que le travail des forces de viscosité est essentiellement lié au cisaillement ($\visc$), pas aux variations de volume (\emph{pas trop} de $\viscbulk$). Une limite $\nu = 0$ est atteinte (pas de variation de volume sous contrainte), auquel cas $\viscbulk = \frac{2\visc}{3}$. On conserve ainsi une composante visqueuse volumique isotrope en $\viscbulk \divergence \vv$, sous contrôle de $\divergence \vv = 0$.

%En terme de nombre de Deborah (\cite{Reiner1964}), rapport entre temps de relaxation de la perturbation et temps d'observation associé à l'échelle macroscopique la plus petite, le cas étudié se situe dans la limite $De = 0$. Elle correspond au cas newtonien, visqueux (faiblement, ici !), associé à un temps de relaxation de la perturbation nul, sous les hypothèses faites (Stokes + Boussinesq).

%\subsection{Réflexions et questions :)}

%%Déjà, à ce stade, j'ai noté que quand on décompose la contrainte entre la partie isotrope $\pression$ et la partie déviatorique (viscosité), on a beau dire qu'on a un fluide visqueux, en fait il est visco-élastique :) On a juste caché dans $\pression$ les contraintes résultantes de la pression/compression isotrope. Ça me semble ok du fait qu'on applique une approximation anélastique qui filtre les ondes de compression/dépression, mais souvent c'est vendu comme allant de soi dans les bouquins que j'ai pu lire. C'est assez trompeur…
%%\plop
%Un truc qui me gêne et avec lequel je n'arrive pas à m'accorder, compte-tenu de mes (faibles :)) connaissances actuelles. Annuler la viscosité de volume,
%\plop
%Dans mon esprit, si on suit un modèle « fluide de Stokes » mais sans le prendre au pied de la lettre, on cherche à dire que le travail des forces de viscosité est essentiellement lié au cisaillement (on mise sur $\visc$), pas aux variations de volume (on ne veut \emph{pas trop} de $\viscbulk$). C'est différent de dire que $\viscbulk = 0$ brutalement : ça signifie en fait qu'on vise une limite $\nu = 0$ (pas de variation de volume sous contrainte uniaxiale), auquel cas $\viscbulk = \frac{2\visc}{3}$ et on conserve une composante visqueuse volumique isotrope en $\viscbulk \divergence \vv$. Je ne suis personne pour remettre en question les approximations établies :) c'est juste que dans l'état de mes connaissances et en suivant « bêtement » mes calculs (que j'espère justes, mais…), quand je lis « $\viscbulk = 0$ définit le fluide de Stokes », j'arrive toujours à la conclusion que c'est bancal d'un point de vue physique (pas numérique…) et ce n'est pas le fait que $\viscbulk$ soit nul qui permet de négliger la viscosité de volume, mais la non-divergence sous l'approximation de Boussinesq (dans le cas présent). Dans le bouquin de Rieutord sur la dynamique des fluides, il est écrit : « en utilisant une approche statistique avec l'équation de Boltzman, on peut montrer que ce cœfficient [$\viscbulk$] est nul (du moins dans l'approximation de Boltzman) pour les gazs monoatomiques. Très souvent, on néglige purement et simplement $\viscbulk$ et lorsqu'on parle de la viscosité dynamique d'un fluide sans plus de précision, on entend $\visc$. Cette approximation qui consiste à négliger $\viscbulk$ s'appelle l'hypothèse de Stokes. On voit qu'en théorie elle ne convient bien qu'aux gazs monoatomiques. » Si tu as un avis sur la question, je suis preneur !
%\plop
%À partir de là, en me basant sur mes calculs rhéologiques :
%\begin{itemize}
%\item soit $\viscbulk \divergence \vv$ s'annule parce qu'on enchaîne sur l'approximation de Boussinesq qui impose $\divergence \vv = 0$, ce qui \emph{certes} revient analytiquement au même que d'avoir posé $\viscbulk = 0$ dès le début, mais à la mérite d'être cohérent avec un $\nu = 0$ plus « réaliste » qu'un cas auxétique (selon moi !) et évite de faire une hypothèse ou une autre sur la valeur de $\visc$ : ça n'impose que $\lambda = 0$, \emph{i.e.} l'incompressibilité. Ça à le mérite d'être assez cohérent ;
%\item soit on peut aussi conserver cette composante en $\divergence \vv$, car j'ai cru comprendre qu'en numérique, avoir $\divergence \vv = 0$ est une gageure, du fait des erreurs d'arrondis. Ça peut alors servir de critère pour surveiller la divergence du modèle (dixit Laëtitia Le Pourhiet ^^) : tant que le $\viscbulk \divergence \vv$ « visqueux » reste petit, on est bon. Sinon, il faut mettre en place un \emph{penality factor} pour corriger l'algorithme en cours de route (je ne sais pas faire là comme ça, mais ça a l'air marrant :)). \plop
%\end{itemize}

%Au final, voilà ce que j'en ai conclu avant de t'écrire ce petit document, pour te demander ton avis éclairé :
%\begin{itemize}
%\item on souhaite éliminer la composante de viscosité de volume $\viscbulk$ et ne conserver que la viscosité en cisaillement $\visc$ ;
%\item l'élasticité est en fait cachée dans la pression hydrostatique $P$, en cohérence avec…
%\item …l'approximation de Boussinesq qui filtre les ondes sonores (anélasticité) et qui permet de justifier mathématiquement pourquoi on en arrive de toute manière à négliger la viscosité en volume de façon élégante ;
%\item ça donne un fluide de Stokes au sens physique du terme, bien que je ne sois pas vraiment d'accord (mais ça vient peut-être de mon incompréhension totale du modèle !) avec l'assomption $\viscbulk = 0$ qui lui est associée dans la littérature (pour moi, ça donne un corps auxétique, bof). Par ailleurs, en faisant $\viscbulk = 0$, je ne trouve pas pareil que dans le texte sur la viscosité que tu m'as donné ($\frac{5}{3}$ au lieu de $\frac{4}{3}$), et je ne vois pas comment on peut trouver cette valeur sans choisir la valeur de $\lambda$, pas de $\viscbulk = \viscbulkexpansion$ ;
%\item à $\visc$ ($==$ en cisaillement) constante, il ne reste qu'un simple laplacien de la vitesse dans les équations — et comme on a les valeurs numériques de la viscosité en utilisant un modèle d'atmosphère (USSA76, NRLMSISE-00...), je crois comprendre que ça ne doit pas être gênant de poser nos calculs à $\visc$ constante, car les pas d'intégration sont petits devant l'échelle spatiale de variation de $\visc$. Utiliser la forme en laplacien me semble dès lors plus sympa que la forme en rotationnel du rotationnel et gradient de la divergence, non ?
%\item en terme de nombre de Deborah, on est au final dans une limite $De = 0$ qui correspond au cas uniquement visqueux, avec un temps de relaxation de la perturbation nul, sous les hypothèses faites (Stokes + Boussinesq).
%\end{itemize}

\subsection{Propagation visqueuse}

La réécriture du système S (\ref{sys:S}) en ajoutant la composante visqueuse sous la forme d'un laplacien du champ de vitesse fait apparaître une dérivée verticale seconde de $\vv$. Contrairement au cas non visqueux, dans lequel toutes les dérivées premières des composantes moyennes de la vitesse étaient advectées par une composante de la perturbation, cette dérivée seconde est « autonome. » En appliquant la même décomposition des $X \in \{ \vv = (u, v, w), \rho, \pression \}$ en la somme d'une partie équilibre $X_0(z)$ et d'une perturbation $X_1(x,y,z,t)$, le passage dans le domaine spectral injecte des facteurs $\neip$ pour ces termes du second ordre, ce qui traduit l'atténuation de la propagation moyenne sous l'effet dissipatif et qui affecte en retour l'évolution de la perturbation.
%Système de départ en se basant sur les calculs ci-avant :
%\begin{subnumcases}{\label{eq:systeme-IGW}(S)}
%\delt \vv + \vv \cdot \gradient \vv = - \frac{1}{\rho} \gradient \pression + \vect g + \nu \laplacien \vv \\
%\divergence \vv = 0 \\
%\delt \rho + \divergence{\rho \vv} = 0
%\end{subnumcases}
%\begin{subnumcases}{(S)}
%\delt \uxp + \uxc \delx \uxp + \uyc \dely \uxp + \uzp \delz \uxc =
%- \frac{1}{\rho_0} \delx \pp + \nu \left [ \left ( \delxs{} + \delys{} + \delzs{} \right ) \uxp + \dzs \uxc \right ] \notag \\
%\delt \uyp + \uxc \delx \uyp + \uyc \dely \uyp + \uzp \delz \uyc =
%- \frac{1}{\rho_0} \dely \pp + \nu \left [ \left ( \delxs{} + \delys{} + \delzs{} \right ) \uyp + \dzs \uyc \right ] \notag \\
%\delt \uzp + \uxc \delx \uzp + \uyc \dely \uzp =
%- \frac{1}{\rho_0} \delz \pp - \frac{\rho_1}{\rho_0} g + \nu \left [ \left ( \delxs{} + \delys{} + \delzs{} \right ) \uzp \right ] \notag \\
%\delx \uxp + \dely \uyp + \delz \uzp = 0 \hfill \notag \\
%\delt{\rho_1} + \uxc \delx{\rho_1} + \uyc \dely{\rho_1} + \uzp \delz{\rho_0} = 0 \notag
%\end{subnumcases}
%On cherche la solution de l'équation d'ondes en prenant $X_1(x,y,z,t) = \tilde X(z) {\mathrm e^{i(\overbrace{k_x x + k_y y - \omega t}^{\Phi})}}$ :
%\begin{subnumcases}{(S)}
%- i \omega \tux + \uxc i k_x \tux + \uyc i k_y \tux + \tuz \dz \uxc = - \frac{1}{\rho_0} i k_x \tp + \nu \left [ \left ( - k_x^2 - k_y^2 \right ) \tux + \dzs \tux + \neip \dzs \uxc \right ] \label{eq:igw-un} \\
%- i \omega \tuy + \uxc i k_x \tuy + \uyc i k_y \tuy + \tuz \dz \uyc = - \frac{1}{\rho_0} i k_y \tp + \nu \left [ \left ( - k_x^2 - k_y^2 \right ) \tux + \dzs \tuy + \neip \dzs \uyc \right ] \label{eq:igw-deux} \\
%- i \omega \tuz + \uxc i k_x \tuz + \uyc i k_y \tuz = - \frac{1}{\rho_0} \dz \tp - \frac{\trho}{\rho_0} g + \nu \left [ \left ( - k_x^2 - k_y^2 \right ) \tuz + \dzs \tuz \right ] \label{eq:igw-trois} \\
%i k_x \tux + i k_y \tuy + \dz \tuz = 0 \label{eq:igw-quatre} \\
%- i \omega \trho + \uxc i k_x \trho + \uyc i k_y \trho + \tuz \dz{\rho_0} = 0 \label{eq:igw-cinq}
%\end{subnumcases}
\begin{subnumcases}{(S)}
i \Omega \tux - \tuz \dz \uxc - \frac{1}{\rho_0} i k_x \tp + \nu \left [ \left ( - k_x^2 - k_y^2 \right ) \tux + \dzs \tux + \neip \dzs \uxc \right ] = 0   \label{eq:igw-un}       \\
i \Omega \tuy - \tuz \dz \uyc - \frac{1}{\rho_0} i k_y \tp + \nu \left [ \left ( - k_x^2 - k_y^2 \right ) \tux + \dzs \tuy + \neip \dzs \uyc \right ] = 0   \label{eq:igw-deux}     \\
i \Omega \tuz - \frac{1}{\rho_0} \dz \tp - \frac{\trho}{\rho_0} g + \nu \left [ \left ( - k_x^2 - k_y^2 \right ) \tuz + \dzs \tuz \right ] = 0              \label{eq:igw-trois}    \\
i k_x \tux + i k_y \tuy + \dz \tuz = 0                                                                                                                      \label{eq:igw-quatre} \\
- i \omega \trho + \uxc i k_x \trho + \uyc i k_y \trho + \tuz \dz{\rho_0} = 0                                                                               \label{eq:igw-cinq}
\end{subnumcases}
avec toujours $\Omega = \omega - \uxc k_x - \uyc k_y$.\\

Avant l'introduction de la viscosité, le propagateur dérivé du système était de la forme $\dz V = A V$. Obtenir une expression similaire n'est plus possible en raison de la présence des termes du second ordre. Il est par contre possible d'écrire un propagateur composite, somme d'un propagateur du premier ordre et d'un propagateur du second ordre dépendant du précédent. Numériquement parlant, l'approche immédiate (cf. \hyperref{annexe:B}{}{}{annexe B}) consiste à appliquer la méthode de résolution présentée pour l'effet des vents, en recalculant les composantes de $K$ et les seconds membres. Le champ de la perturbation de vitesse horizontale apparaît comme inconnue explicite dans les formulation des variables normalisées $\tuz^*$ et $\tp^*$ et doit donc être ajoutée au vecteur solution $U$, ce qui multiplie par deux la dimension du système en 3D. Des différences finies centrées d'ordre 2 sont introduites, si bien que les conditions initiales sont maintenant doublées afin d'initier la résolution au second ordre par une propagation non visqueuse.
Les calculs peuvent être effectués à $\visc$ constante tant que les pas d'intégration sont petits devant l'échelle spatiale de variation de $\visc$.

Cette méthode directe échoue car le système diverge très rapidement. La raison est en que certaines composantes de $K$ et de $B$ sont très proches de 0 au regard de la précision numérique disponible, de sorte que le système est fortement mal conditionné, avec notamment $K$ proche d'une matrice singulière (à déterminant nulle, \ie. non inversible).

Pour augmenter le taux de convergence vers une solution correcte, deux approches sont possibles :
\begin{itemize}
    \item au niveau de la dérivation analytique, adimensionner S en définissant des échelles caractéristiques pour les variables indépendantes et dépendantes.
    \item au niveau de la discrétisation, préconditionner $K$, afin de lui conférer des propriétés « spectrales » plus favorables à une résolution vectorielle. On souhaite alors se donner un préconditionneur $G$ qui, appliqué au système linéaire, permette de le ramener dans un domaine numériquement acceptable. Cette approche peut être utilisé en complément de l'adimensionnement, car le traitement spectral du système modifie les ordres de grandeurs.
\end{itemize}

À l'issue du stage, l'implémentation d'un code fonctionnel basé sur le système adimensionné n'était pas terminée, aussi l'étude numérique de l'effet de la viscosité n'a pas été menée à son terme. Toutefois, des considérations analytiques et numériques sont présentées ci-après.
%En fait, compte-tenu du pas spatial et du modèle atmosphérique utilisé, la dissipation visqueuse est attendue non négligeable à partir d'une hauteur de plusieurs dizaines de kilomètres. Pour le vérifier, on aurait pu implémenter une construction de $K$ en deux parties, en laissant le choix du seuil en altitude à partir duquel la propagation visqueuse prend le relais du cas non visqueux, et plotter l'écart relatif au cas non visqueux en fonction de z.

%Dans l'équation \refeq{eq:igw-quatre}, on isole le $\dz \tuz$ et on exprime les autres termes avec \refeq{eq:igw-un} et \refeq{eq:igw-deux} (on peut noter $\Omega = \omega - \uxc k_x - \uyc k_y$). Ça donne un des deux termes de V. Désormais, il y a dans le membre de droite des trois premières équations du $\dzs \tui$. Avec \refeq{eq:igw-trois} et \refeq{eq:igw-cinq}, on construit $\dz \tp$ en fonction de $\dz \tuz$.
%\plop
%Dans l'idée, le système linéaire se résout pareil que dans le cas non-visqueux, en différences finies. Je pense qu'il est sage de les calculer centrées, aux deux ordres, mais faut voir pour les conditions initiales. \emph{Todo asap} :)

\subsection{Adimensionnement, potentiel dissipatif de la viscosité}

% TODO
% partie précédente : décrire le raccordement sqrt(gh) !
% justifier l'échelle de longueur verticale !
% écrire le système adimensionné
% parler de la loi de similitude, Reynolds
% faire l'analyse en ordre de grandeur
% conclusion : le code visqueux ne le sera que partiellement, pour gain d'efficacité

Afin de réaliser l'adimensionnement du système étudié, il est nécessaire d'introduire des échelles typiques de temps (dynamique) et de longueur pour les variables indépendantes, de vitesse, de pression et de densité pour les variables dépendantes. Dans le cadre d'une étude modale, le choix de ces échelles est guidé par les caractéristiques spectrales du système. Une échelle typique de temps sera déduite de la période étudiée ; lui sera couplée l'échelle typique de longueur horizontale \emph{via} la longueur d'onde étudiée (choisie ou imposée par le raccordement IGW tsunamigénique -- IGW atmosphérique). On considèrera donc les paramètres en entrée du système (au sol). L'échelle typique de longueur verticale est moins contrainte (du fait de la forme explicite en $\tuz^*$) ; on la prendra égale à l'échelle horizontale pour des raisons numériques. Les échelles typiques de vitesses seront celles des vitesses au sol. Ainsi :
\begin{align}
u    & = U u^*, v = V v^*, w = W w^* \textrm{~avec~} U = V = W \\ \notag
x    & = L x^*, y = L y^*, z = L z^* \\ \notag
t    & = \frac{L}{U} t^*, P = \rho U^2 P^*, \rho = R \rho^* \notag
\end{align}
\begin{subnumcases}{(S)}
    \deltet{\vv^*} + \lp \vv^* \cdot \gradient \rp \vv^* & $= -\gradient P^* + \frac{\mu}{\rho U L} \laplacien \vv^*$ \\
    \divergence \vv^* & $= 0$ \\
    \deltet{\rho^*} + \divergence \lp \rho^* \vv^* \rp & $= 0$
\end{subnumcases}

On isole ainsi le nombre de Reynolds $Re = \Reynolds$ qui demeure le seul paramètre des équations du mouvements. En prenant 100 km comme ordre de grandeur moyen de la longueur d'onde et 1 pour la masse volumique, on a $Re = 10^{10}U$. Alors $Re \leq 1 \equivaut U \leq 10^{-10}$ ; on peut donc s'attendre \emph{a priori} à ce que la dissipation visqueuse soit négligeable. Toutefois, le profil de densité atmosphérique entre en jeu. Au sol, $\rho$ est de l'ordre de l'unité. Par exemple, vers 130 km d'altitude, le modèle USSA76 donne une valeur de $10^{-9}~\textrm{kg}.\textrm{m}^{-3}$, ce qui correspond à $Re = 10U$ \ie $U \leq 0,1~\textrm{m}.\textrm{s}^{-1}$. À partir d'une certaine altitude, selon l'ordre de grandeur des perturbations, il est attendu que la viscosité joue un rôle dissipatif non négligeable.

\subsection{Approche numérique}

Même adimensionné, un système n'est pas nécessairement conditionné de façon optimale. Dans le cas présent, le passage dans l'espace fréquentiel introduit des combinaisons de cœfficients de faible ordre de grandeurs. Un préconditionnement numérique peut être utile pour construire une méthode de résolution stable et extensible à d'autres paramètres (auquel cas l'adimensionnement analytique devient plus délicat).

Le principe général d'un préconditionnement pour un système linéaire $KU = B$ est de trouver $G$, approximation de $K^{-1}$, de sorte que $GKU = GB$ soit un système linéaire équivalent, à la précision numérique près, au système de départ. Évidemment, $G$ doit pouvoir être construit et appliqué rapidement. Si $G$ est une approximation directe de $A^{-1}$, le préconditionnement est explicite ; si $G = M^{-1}$ avec $M$ une approximation creuse de $A$, le préconditionnement est implicite et son application nécessite la résolution de systèmes linéaires dérivés.
%Déterminer $G$ n'est pas trivial et on peut-être amené à utiliser une des deux approches classiques : un préconditionneur itératif, basé sur un algorithmé généraliste « tout terrain » efficace mais non optimal, ou un préconditionneur ponctuel, déduit des propriétés spécifiques du système (symétrie, densité et magnitudes des valeurs, etc.) La seconde approche peut se révéler plus efficace, surtout s'il est possible de partir d'un algorithme itératif puissant, mais c'est un travail complexe.

Dans le problème étudié, les propriétes de $K$ amènent à penser qu'un préconditionnement explicite serait plus judicieux, car $K$ n'est pas symétrique : la plupart des méthodes de préconditionnement classiques, implicites, ne peuvent s'appliquer qu'à des matrices symétriques définies positives (SPD pour \emph{symmetric positive definite}) \ie symétrique et dont toutes les valeurs propres sont strictement positives. Par ailleurs, dans la perspective d'une parallélisation du code, l'application explicite présente une meilleure atomicité et réduit les besoins de communications. Les problèmes de convection-diffusion en particulier génèrent des systèmes linéaires fortement creux et non symétriques (NSPD) ; toutefois, des méthodes implicites plus généralistes, rapides, ont été récemment développées pour gérer les systèmes non symétriques (\cite{Rezghi2006}, \cite{Benzi1998}).

$K$ est en effet une matrice fortement creuse : un préconditionneur explicite efficace $G$ devrait dans l'idéal l'être dans les mêmes proportions. Sous ces conditions et en utilisant une méthode bien choisie, par exemple du type gradient conjugé, l'application de G au système pourra se limiter à de simples multiplications matrices-vecteurs. Toutefois, une approximation directe de $K^{-1}$ n'est pas nécessairement plus efficace qu'une approximation indirecte, car $K$ présente un caractère fortement diagonal qui se prête particulièreme bien à l'application des méthodes implicites, du type factorisation LU. Somme toute, les deux approches semblent judicieuses.

