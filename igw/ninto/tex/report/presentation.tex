\section{Présentation}

\subsection{Contexte du stage}

Ce stage a été effectué au sein de l’équipe \hyperref{http://ganymede.ipgp.jussieu.fr}{}{}{Études spatiales et planétologie} (IPGP UMR 7154 -- CNRS -- Université Paris 7) dirigée par Philippe Lognonné, sous la direction de Giovanni Occhipinti. Les travaux de l'équipe sont rattachées à la planétologie et à l'observation de la Terre, notamment pour caractériser les mécanismes de couplage enveloppes solides--océan--atmosphère. L'étude des planètes telluriques sous l'angle de la télédétection et de la géophysique interne sont au cœur de l'activité de recherche fondamentale, avec de nombreuses applications pour des missions spatiales et d'observation. Le présent stage s'insérait dans la large thématique des couplages terre interne--atmosphère, \emph{via} la caractérisation des perturbations atmosphériques transitoires associés aux évènements sismiques.

\subsection{Les ondes de gravité internes atmosphériques}

Le travail effectué lors de ce stage a été consacré une étude paramétrique des ondes de gravité internes (IGW, \emph{internal gravity waves}) atmosphériques, limitée à l'effet des vents et de la viscosité. Les IGW constituent la réponse à une perturbation d'un milieu fluide stratifié, généralement verticalement sous l'effet d'un champ de pesanteur et pour lequel la force d'Archimède joue le rôle de force de rappel — océan et atmosphère répondent en particulier à cette définition. Cette structuration verticale impose aux IGW une propagation tributaire de divers paramètres influençant, à différentes échelles, la stratification et donc le milieu de propagation des ondes, ainsi que les ondes elle-mêmes. On pourra ainsi avoir pour objectif de caractériser l'influence de phénomènes dissipatifs, d'interactions et de couplages, et ce dans les différentes couches atmosphériques.

Eu égard à leur rôle important dans l'évolution de la circulation et de la structure atmosphérique, notamment aux grandes échelles, les IGW constituent un sujet d'étude relativement récent. Dans une synthèse sur les travaux des deux dernières décennies, \cite{Fritts2003} insistent sur les progrès réalisés notamment grâce à l'avènement de la télédection et d'une modélisation numérique puissante : propagation verticale, phénomènes d'interactions et de turbulence, caractérisation spectrale…

Par conservation de l'énergie cinétique, la propagation des IGW atmosphériques donne lieu à une amplification — qui peut être très importante — proportionnelle à l'inverse de la racine carrée de la densité atmosphérique. Le phènomène se réalise sous contrainte du milieu local de propagation (atmosphère libre, ionisée, régime des vents, gradients de température, de vorticité, etc.) La propagation affecte le milieu de propagation, par exemple en générant de la turbulence et par dépôt d'énergie (phénomènes dissipatifs). D'un point de vue expérimental, la signature des IGW est plus particulièrement mise en évidence, par télédétection, à travers la perturbation qu'elle induit dans la très haute atmosphère ionisée (ionosphère) où se réalise un couplage magnétique. Cette perturbation dépend donc en partie des évènements ayant eu lieu lors de la propagation dans l'atmosphère neutre.

Le stage proposait de réaliser une étude de l'effet des vents et de la viscosité sur la propagation des IGW dans une atmosphère neutre réaliste (modèle \emph{U.S. Standard Atmosphere}, USSA76), indépendamment d'un couplage au champ magnétique terrestre dans l'atmosphère ionisée. Un travail analytique préparatoire devait mener à une implémentation numérique. Les IGW sont plus particulièrement au cœur de la « sismologie atmosphérique » qui vise à caractériser la signature atmosphérique des évènements sismiques, qu'ils aient été générés dans les structures internes ou dans l'enveloppe gazeuse. Plus particulièrement, il s'agissait ici de pouvoir caractériser des IGW tsunamigéniques intervenant dans le couplage sismique terre solide -- océan -- atmosphère. Une caractérisation fine de la propagation des ondes liées aux évènements sismiques océaniques est la clé pour la résolution de problèmes inverses et la prévention sismique.

%Du fait de la conservation de l’énergie cinétique et de la diminution de la densité atmosphérique en
%altitude (‫= ݇ܧ‬    ݉‫ ,)2ݒ‬l’atmosphère joue le rôle d’amplificateur naturel de la vitesse verticale (au sol)
                %ଵ
                %ଶ
                                                                %5
%qui est alors multipliée par un facteur pouvant aller jusqu’à 10 lorsque l’onde atteint l’ionosphère. Un
%des objectifs des nouveaux ‘sismologues atmosphériques’ est de pouvoir détecter cette onde via des
%satellites afin d’augmenter la densité des données sismiques et, dans le cas des tsunamis, de
%développer des nouveaux systèmes d’alertes préventives.


% TODO il faut parler de 1. milieu de propagation 2. milieu de détection avec des comportements et couplages différents dans les deux domaines atmosphériques

% nouvelle partie, sur la viscosité :

% copcol plus ou moins l'autre .tex, mais le réorganiser :
% petite intro/problématique (viscoisté négligeable d'aprèg Fritts…), puis
% ma dérivation analytique
% et ma tentative numérique, et pourquoi ça a foiré à mon avis

% en conclusion, ben… résumé plus ouverture sur ce que je vais continuer à faire


% --------------------------------------------------------------------------
%- pourquoi les étudier ?
%- quels sont les résultats généraux utiles ?
%- sur quels travaux précis se base mon travail (Sun…) ?
%- étude de l'effet des vents et de la viscosité : une partie analytique, une partie numérique
%- quels résultats ?
%- quelles perspectives ?
